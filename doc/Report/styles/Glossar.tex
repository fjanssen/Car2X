\chapter{Glossar}

\begin{description}[\breaklabel \setleftmargin{10pt}]

\item[Debuggen] Auffinden und Beseitigen von Softwarefehlern.

\item[\ac{ECU}] Elektronische Komponente die vor Allem in Eingebetteten Systemen eingesetzt werden, um Regelungs- und/oder Steuerungsaufgaben zu �bernehmen. 

\item[Debugger] Software zum Debuggen von ECUs(z.B. GDB).

\item[Flasher] Software zum Programmieren von ECUs, vornehmlich ben�tigt f�r das
Roll-Out (z.B. Nexus, Trace32, LuminaryMicro-Programmer, etc.).

\item[Flashen] Programmiern einer ECU.

\item[Proxy] Software die eine Verbindung zwischen Debugger/Flasher und dem
Programmer herstellt, sorgt gew�hnlich f�r die Umsetzung von TCP/IP nach
USB (z.B. OpenOCD).

\item[Target] Bezeichnung f�r die Zielplattform auf der die Anwendung laufen soll. 

\item[Host] Bezeichnung f�r die Entwicklungsplattform der Anwendung.

\item[\ac{JTAG}] IEEE-Standard 1149.1 zum Testen und Debuggen von ECUs direkt in der Schaltung. Dieses Verfahren ist auch als Boundary Scan Test bekannt.

\item[Programer] Hardware das JTAG-Interface zur Verf�gung stellt, sorgt f�r
die Umsetzung von USB nach JTAG (z.B. Lauterbach, ARM-USB-OCD). Ein Programer wird dazu genutzt die Software die in Bin�r-Dateien vorliegt auf die Zielplatform zu programmieren.

\item[On-Target Debugger] Hardware, die �ber eine Schnittstelle die CPU des Targets ansteuern kann. Ein On-Target Debugger bietet h�ugig die folgenden Funktionali�ten an: Starten und Anhalten der CPU, Auslesen und Schreiben von Registern. H�ufig sind die Funktionalit�ten eines Programers in einem On-Target Debugger integriert.

\item[In-Circuit Debugger] Hardware, die sich auf dem Target befindet und die gleiche Funktionalit�t anbietet wie ein On-Target Debugger.

\end{description}