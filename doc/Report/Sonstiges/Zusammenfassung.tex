%\clearemptydoublepage






\vspace*{2cm}
\begin{center}
{\Large \bf Zusammenfassung}
\end{center}
\vspace{1cm}


Auf Grund der zunehmenden Benutzung des modellbasierten Entwicklungsprozesses, steigt der Bedarf nach ad�quaten Konzepten zum Debugging von Modellen. Die Softwareentwicklung findet zwar auf Modell-Ebene statt, aber das Debugging wird auf der Sourcecode-Ebene durchgef�hrt. Es fehlen h�ufig die M�glichkeiten, die verwendeten Beschreibungstechniken f�r Modelle, wie beispielsweise Systemstrukturdiagramme oder Zustandsautomaten, zum Debugging zu verwenden. Das Debugging von Source- oder Maschinencode ist bei einem modellbasierten Entwicklungsprozess unpassend. Das bei der modellbasierten Entwicklung gewonnene Abstraktionsniveau, geht beim Debugging auf der Sourcecode-Ebene wieder verloren. Um das Abstraktionsniveau beim Debugging beibehalten zu k�nnen, wurden in dieser Ausarbeitung die Konzepte des modellbasierten Debuggings ausgearbeitet und in einem Werkzeugprototypen umgesetzt.


\newpage