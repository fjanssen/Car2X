\section{Workflow and guidance}

\subsection{Nios2 workflow}

The Nios2 processors are programmed similar but always a little bit different. The program depends on the corresponding CMW-Unit and especially on the connected sensors. For example the left motors have to move in the opposite direction than the right ones to let the car move forward. Another example is that only two FPGA boards are connected to the ultrasound sensors.\\

The differences are collected in the 'motor\_control/properties.h' file. You may have to change the following constants:\\

\begin{tabular}{l | p{11.75cm}}
INVERTED & One side has to be 1 and the the other side 2.\\ \hline
AVAILABLE\_OPERATIONS & Array which contains the (message-)types of all messages which can be understood by this build. Only those types are included that are optional (so said with a value $>=$ 8). \\
 & Currently only max. 4 message-types are supported. All 4 array elements have to be filled with a value. Are there less then 4 messages the rest should be filled up with 0x00.
\end{tabular}

\subsubsection{Programming the device}
You may use this workflow to program the Nios2 processors:
\begin{enumerate}
	\item Start the NIOS II Eclipse Plug-in
	\item Connect all batteries.
	\item Connect one of the DE0-nano boards with the USB cable to your laptop.
	\item Change INVERTED to the necessary value.
	\item What sensors are connected to this FPGA? Don't forget the sensors which are directly on the board (such as GSensor).
	\item Look up the message types of the collected sensors.
	\item Fill up AVAILABLE\_OPERATIONS with this message types. Please note that only those types have to filled in with types $>=$ 8.
	\item Generate and load the .elf file to the FPGA.
	\item Continue with the other FPGAs at step 3.
	\item After you loaded all four FPGAs turn the car head down. Start the central ECU.
	\item The central-ECU sends the WelcomeMessage and MotorMeasurementMessage. The wheels will start tuning with full speed (thus the head down).\\
\end{enumerate}

Maybe the advisers can activate the flash memory to start the motor program on power-up automatically!

\subsubsection{Tour through program files}

The program files are organized as the following tree structure:
\begin{itemize}
	\item \textit{docu}: the documentation incl. the presentations
	\item \textit{firmware}: all software files
	\begin{itemize}
		\item \textit{main}: the 'main' branch of the files
		\item \textit{test}: files and programs for testing
		\item \textit{tools}: tools such as code-generators, ...
	\end{itemize}
	\item \textit{hardware}
	\item \textit{etc}	\\
\end{itemize}

The main folder itself contains the following structure:
\begin{itemize}
	\item \textit{application}: the projects and make-files
	\item \textit{car\_control\_linux}: main files for the central processor (Linux-PC)
	\item \textit{car\_control}: a little outdated version of above but for Windows
	\item \textit{etc}: utility libs
	\item \textit{external\_drivers}: sensor and communication drivers
	\item \textit{motor\_control}: main files for the motor ECUs
	\item \textit{networking}: CarProtocol, CarMessages and derived messages for the sensors
	\item \textit{sensors}: files which represent the state of a sensor
	\item \textit{terasic\_lib}: library files for the DE0-Nano board
	\item \textit{web\_interface}: code and content files for web services\\
\end{itemize}

\subsubsection{Guide to the ComBuilder}

You can find the guide to the ComBuilder code-generator under \textit{firmware/tools/ComBuilder/docu}.

\subsection{Central ECU workflow} 

\begin{enumerate}
\item The central ECU boots the Debian OS from a SD card. This may take a minute or two.
\item After booting the system opens a Wi-Fi-Accesspoint named 'SafetyCarNet'. You are now able to connect to this network.\\
Passphrase: safetycarnet
\item The dhcp service takes very long (1 - 2 minutes) to give you an IP address. Please check your system IP  if you are really connected to the network!
\item Use a SSH client to connect to the terminal. A useful client for Windows is 'WinSCP'.\\
Rootname: root \hspace{2em} Password: root\\
Username: board \hspace{2em} Password: board
\item copy the changed code files to the corresponding location (under root/SafetyCar/firmware/main/).
\item change current directory to root/SafetyCar/firmware/main/application/car\_control\_linux.
\item build program with: make
\item check build result
\item start program with: ./car\_control\\
\end{enumerate}
  
There is also an Apache-Webserver running in background. The web-files are under the standard path for this server type in the folder www.

\subsection{Maintaining the Ethernet-to-UART chips}

The Ethernet-to-UART chips have the following addresses:\\
192.168.0.11, 192.168.0.12, 192.168.0.13, 192.168.0.14\\

Just connect your laptop via a LAN-cable to the switch and change your local IP(v4) address to 192.168.0.200 (subnet: 255.255.255.0). You can now enter one of the addresses above in your browser and get the config-webpage of the corresponding chip. See the manual of the chips under docu!

\subsection{Additional help}

If you are in need of additional help, feel free to contact me under florian.hisch@tum.de!