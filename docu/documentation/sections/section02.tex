\clearpage
\section{Construction of the car}

First we didn't know which dimensions to take for the car. We had to decide between a big but heavy plank and a smaller one. We played around with different sizes and chose a wooden board with about 50 cm x 40 cm first. But afterwards we thought this is too big, so we optimized the size.\\
Now this wooden panel is about 40 cm x 35 cm. This dimension has barely enough space for many sensors (camera, ...) but is sufficiently light to give more dynamic driving behavior to the car. Even though the car is not as dynamic as a little racing model and behaves more like a little tank.\\

After that we wanted to fix the motors on this wooden panel. We took four cubic blocks of wood as spacer between the plank and the motor. Doing this gives us enough space to fix the batteries on the bottom side of the plank. We screwed the said blocks in each corner of the panel and on each block the motors. An additional advantage is the possibility to make the motors steerable.\\

Every motor was measured about his dynamic and static behavior. We distributed the motors int a way that a 'slower' can be compensated by a stronger one. Especially the difference in maximum speed is clearly visible in a case of a wrong distribution. Fixing the four wheels on the motors was in contrast very easy. Therefor we only had to fix one screw.\\

Then we considered the distribution of the remaining components. We kept an eye on not loosing too much stability with the many bore holes. We placed the H-bridges in the corners due to the needed air cooling. Next to each H-bridge is the corresponding FPGA board. During the development process the interconnection between those boards changed several times. We are still not happy about the stability of the used cables. Maybe someone can make the connectors of the cables stronger.\\

The batteries provide around 11.5 Volts but most of the electronic runs on 5 Volts. In the first version there were four linear voltage converters one for each FPGA. Those small components are barely strong enough to handle currents about one Ampere. Since one FPGA board needs at least one Ampere and we haven't talked about the H-bridges by now, the voltage converters became very hot. This made our decision to place the power supply in the middle of the panel not very well thought-out. The DC/DC converters, which shipped in in February, solved that problem.\\

So there was not any more space on the panel. That's why we needed to make one more floor. For that we decided to use steel sticks as spacers. For the second floor we also took a wooden panel. But this one is much smaller. It isn't as thick as the panel of the ground floor and is so lighter. The size is about 20 cm x 25 cm.\\

From the bottom up we fixed the LAN switch on the second wooden panel. There was no possibility to screw it so we had to glue it. The LAN switch is connected to the four Ethernet-UART-bridges via LAN cables. These given cables were too long so we contracted them. The four Ethernet-UART-bridges are also glued at the smaller wooden plank from the bottom up.\\

On the top of the smaller panel is the central ECU board (Linux-PC). It is connected with the LAN switch via one LAN cable. Therefor we also had to bore a hole into this panel.\\

At the end we got two ultrasonic sensors. We fixed them at the ledges of the big wooden board. One ultrasonic sensor is in front of the car and one ultrasonic sensor is at the backside of the car. Each 
is connected with one FPGA board. 