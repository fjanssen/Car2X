\section{Introduction}

\subsection{Outline}
We got the task to design and constructed a car model during our lab course HW/SW-Co-design. The main aim on this was to design a distributed software architecture. This architecture consists of four FPGAs and a central Linux-PC.\\

The car itself has a tower-like structure with different sized levels. These levels are made out of wooden planks which are robust enough but not too heavy in order to the motor power. As pillars between the levels we took metal and wooden sticks.\\
We started with the biggest plank (plywood) and step by step we fixed all components on it. We wanted to assemble the car as fast as possible, but the components shipped in only in small portions. Hence we also developed the software accordingly.\\

Our first aim was to make the car drive, controlled by a keyboard. Then we wanted to make the control more comfortable and start to play with Wi-Fi. At this point we want to say 'Thank You' to our partner team, which helped us with our Linux-PC and the Wi-Fi very well!\\

To get some intelligence into the car control we plan to make our car stopping if something is crossing the car's way. So ultrasonic sensors were fixed at the front and at the end of the car.\\

We also played with a simple 'exploration mode'. The car should be driving in one direction. Is there an obstacle it turns left until the obstacle is not in it's way anymore. Then the car drives further in this direction.

\subsection{Prelude}
Our team exists of Florian Hisch, Christoph Weinisch and Lukas Kunzemann. We all are studying computer science in the fifth semester. The lab course in WS 2013/14 started in the end of October. So we got about three months to work on this project. Even though the course was labeled 'on LEGO-Cars' there was no 'brick-made car' left. Our assignment was instead assembling a 'self-made-car' and to implement all the sensors which are needed to navigate.\\

We got almost all components from our advisers. Only some trifles we brought by ourselves.
